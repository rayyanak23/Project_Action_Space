% Options for packages loaded elsewhere
\PassOptionsToPackage{unicode}{hyperref}
\PassOptionsToPackage{hyphens}{url}
%
\documentclass[
]{article}
\usepackage{amsmath,amssymb}
\usepackage{iftex}
\ifPDFTeX
  \usepackage[T1]{fontenc}
  \usepackage[utf8]{inputenc}
  \usepackage{textcomp} % provide euro and other symbols
\else % if luatex or xetex
  \usepackage{unicode-math} % this also loads fontspec
  \defaultfontfeatures{Scale=MatchLowercase}
  \defaultfontfeatures[\rmfamily]{Ligatures=TeX,Scale=1}
\fi
\usepackage{lmodern}
\ifPDFTeX\else
  % xetex/luatex font selection
\fi
% Use upquote if available, for straight quotes in verbatim environments
\IfFileExists{upquote.sty}{\usepackage{upquote}}{}
\IfFileExists{microtype.sty}{% use microtype if available
  \usepackage[]{microtype}
  \UseMicrotypeSet[protrusion]{basicmath} % disable protrusion for tt fonts
}{}
\makeatletter
\@ifundefined{KOMAClassName}{% if non-KOMA class
  \IfFileExists{parskip.sty}{%
    \usepackage{parskip}
  }{% else
    \setlength{\parindent}{0pt}
    \setlength{\parskip}{6pt plus 2pt minus 1pt}}
}{% if KOMA class
  \KOMAoptions{parskip=half}}
\makeatother
\usepackage{xcolor}
\usepackage[margin=1in]{geometry}
\usepackage{longtable,booktabs,array}
\usepackage{calc} % for calculating minipage widths
% Correct order of tables after \paragraph or \subparagraph
\usepackage{etoolbox}
\makeatletter
\patchcmd\longtable{\par}{\if@noskipsec\mbox{}\fi\par}{}{}
\makeatother
% Allow footnotes in longtable head/foot
\IfFileExists{footnotehyper.sty}{\usepackage{footnotehyper}}{\usepackage{footnote}}
\makesavenoteenv{longtable}
\usepackage{graphicx}
\makeatletter
\def\maxwidth{\ifdim\Gin@nat@width>\linewidth\linewidth\else\Gin@nat@width\fi}
\def\maxheight{\ifdim\Gin@nat@height>\textheight\textheight\else\Gin@nat@height\fi}
\makeatother
% Scale images if necessary, so that they will not overflow the page
% margins by default, and it is still possible to overwrite the defaults
% using explicit options in \includegraphics[width, height, ...]{}
\setkeys{Gin}{width=\maxwidth,height=\maxheight,keepaspectratio}
% Set default figure placement to htbp
\makeatletter
\def\fps@figure{htbp}
\makeatother
\setlength{\emergencystretch}{3em} % prevent overfull lines
\providecommand{\tightlist}{%
  \setlength{\itemsep}{0pt}\setlength{\parskip}{0pt}}
\setcounter{secnumdepth}{-\maxdimen} % remove section numbering
\ifLuaTeX
  \usepackage{selnolig}  % disable illegal ligatures
\fi
\IfFileExists{bookmark.sty}{\usepackage{bookmark}}{\usepackage{hyperref}}
\IfFileExists{xurl.sty}{\usepackage{xurl}}{} % add URL line breaks if available
\urlstyle{same}
\hypersetup{
  pdftitle={ACTION SPACE},
  hidelinks,
  pdfcreator={LaTeX via pandoc}}

\title{ACTION SPACE}
\author{}
\date{\vspace{-2.5em}2024-04-05}

\begin{document}
\maketitle

\hypertarget{a-statistical-data-analysis-on-heart-disease-prediction}{%
\section{A Statistical Data Analysis on Heart Disease
Prediction}\label{a-statistical-data-analysis-on-heart-disease-prediction}}

\hypertarget{abstract}{%
\subsection{Abstract}\label{abstract}}

World Health Organization has estimated 12 million deaths occur
worldwide, every year due to heart diseases. Half the deaths in the
United States and other developed countries are due to cardio vascular
diseases. The early prognosis of cardiovascular diseases can aid in
making decisions on lifestyle changes in high risk patients and turn
reduce the complications. This project intends to pinpoint the most
relevant risk factors of heart disease as well as predict the overall
risk using logistic regression.

\hypertarget{introduction}{%
\section{1 INTRODUCTION}\label{introduction}}

\textbf{1.1 WHAT IS HEART DISEASE:}

\includegraphics[width=3.20833in,height=\textheight]{image1.jpeg}

A type of disease that affects the heart or blood vessels.The term
``Heart disease'' refers several types of heart conditions. The most
common heart disease is coronary artery disease (narrow or blocked
coronary arteries), which can lead to chest pain, heart attacks, or
stroke. Other heart diseases include congestive heart failure, heart
rhythm problems, congenital heart disease (heart disease at birth), and
endocarditis (inflamed inner layer of the heart). Also called
cardiovascular disease. Heart disease is the leading cause of death in
the United States, according to the Centers for Disease Control and
Prevention (CDC). In the United States, 1 in every 4 deaths in is the
result of a heart disease. \textbf{1.2 HOW THE HEART WORKS:} To
understand the causes of heart disease, it helps to understand how the
heart works. heart is a pump. It's a muscular organ about the size of
our fist, located slightly left of center in our chest. Heart is divided
into the right and the left sides.

\begin{itemize}
\tightlist
\item
  The right side of the heart includes the right atrium and ventricle.
  It collects and pumps blood to the lungs through the pulmonary
  arteries.
\item
  The lungs give the blood a new supply of oxygen. The lungs also
  breathe out carbon dioxide, a waste product.
\item
  Oxygen-rich blood then enters the left side of the heart, including
  the left atrium and ventricle.
\item
  The left side of the heart pumps blood through the largest artery in
  the body (aorta) to supply tissues throughout the body with oxygen and
  nutrients.
\end{itemize}

\includegraphics[width=4.98958in,height=\textheight]{image2.jpeg}

FIGURE 1: Cross section of heart

\hypertarget{different-types-of-heart-disease-and-their-causes}{%
\subsubsection{1.3 DIFFERENT TYPES OF HEART DISEASE AND THEIR
CAUSES:}\label{different-types-of-heart-disease-and-their-causes}}

There are many types of heart disease. Some of them include:

\hypertarget{coronary-artery-disease-or-cad}{%
\paragraph{(1) Coronary Artery Disease OR
CAD}\label{coronary-artery-disease-or-cad}}

\hypertarget{section}{%
\paragraph{\texorpdfstring{\protect\includegraphics{image3.jpeg}}{}}\label{section}}

\textbf{Causes:} A buildup of fatty plaques in arteries
(atherosclerosis) is the most common cause of coronary artery
disease.Plaque buildup causes the inside of the arteries to narrow over
time, which can partially or totally block the blood flow. This process
is called atherosclerosis.

\#\#\#(2) Congenital Heart Disease: Congenital heart defects usually
develop while a baby is in the womb. Heart defects can develop as the
heart develops, about a month after conception, changing the flow of
blood in the heart. \textbf{Causes:} Some medical conditions,
medications and genes may play a role in causing heart defects, Drinking
alcohol, having diabetes or having habit of smoking during pregnancy can
defect baby's heart. Heart defects can also develop in adults. As human
ages, heart's structure can change, causing a heart defect.

\includegraphics[width=6.45833in,height=\textheight]{image4.jpeg} FIGURE
2: Congenital heart disease

\hypertarget{heart-attack-or-myocardial-infraction}{%
\paragraph{(3) Heart Attack or Myocardial
Infraction}\label{heart-attack-or-myocardial-infraction}}

Heart attack happens when the arteries leading to the heart become
blocked, disrupting blood flow.

\includegraphics[width=4.03125in,height=\textheight]{image5.jpeg}

\textbf{Causes:} The leading cause of heart attacks is coronary heart
disease. This is usually due to cholesterolcontaining deposits called
plaques. Plaques can narrow the arteries, reducing blood flow to the
heart. If a plaque breaks open, it can cause a blood clot in the heart.
A heart attack may becaused by a complete or partial blockage of a heart
(coronary) artery.

\hypertarget{heart-muscle-disease-or-cardiomyopathy}{%
\subsubsection{(4) Heart Muscle Disease or
Cardiomyopathy:}\label{heart-muscle-disease-or-cardiomyopathy}}

This condition can lead to heart failure. It occurs when the heart
muscle becomes larger and stiffens, preventing it from pumping blood
away from the heart. Some times blood can pool in the lungs.

\textbf{Causes:}

(i)Long-term high blood pressure

(ii)Heart tissue damage from a heart attack

(iii)Long-term rapid heart rate, can lead to Cardiomyopathy

\includegraphics[width=6.9375in,height=\textheight]{image6.jpeg} FIGURE
3: Cardiomyopathy

\hypertarget{heart-failure-or-congestive-heart-failure}{%
\paragraph{(5) Heart Failure or Congestive Heart
Failure}\label{heart-failure-or-congestive-heart-failure}}

This condition occurs when stiffness in the heart prevents the organ
from pumping blood adequately through the body.

\includegraphics[width=4.73958in,height=\textheight]{image7.jpeg}

\textbf{Causes:} Conditions including high blood pressure,valve disease,
thyroid disease, kidney disease, diabetes, or heart defects present at
birth can all cause heart failure. In addition, heart failure can happen
when several diseases or conditions are present at once.

\hypertarget{heart-valve-disease}{%
\paragraph{(6) Heart Valve Disease}\label{heart-valve-disease}}

Valve disease happens when any of the four valves in the heart don't
open or close properly and interrupt blood flow. If the defect in the
valve happens at birth, it's called congenital heart disease.
\textbf{Causes:} of valve disease: History of certain infections that
can affect the heart History of certain forms of heart disease or heart
attack High blood pressure, high cholesterol, diabetes and other heart
disease risk factors Heart valve disease can cause many complications,
including:,Heart failure, Stroke, Blood clots.

\includegraphics[width=5.64583in,height=\textheight]{image8.jpeg}

\hypertarget{abnormal-heart-rhythms-or-arrhythmia}{%
\paragraph{7. Abnormal Heart Rhythms OR
Arrhythmia:}\label{abnormal-heart-rhythms-or-arrhythmia}}

This condition causes a fluctuation in the heartbeat that happens while
at rest. \textbf{Causes:} :Excessive use of alcohol or caffeine,
diabetes, stress, valvular heart disease can causes arrhythmia.

\hypertarget{risk-factors}{%
\subparagraph{1.4 RISK FACTORS:}\label{risk-factors}}

Risk factors for developing heart disease include:

\begin{itemize}
\tightlist
\item
  \textbf{Age:} Growing older increases the risk of damaged and narrowed
  arteries and aweakened or thickened heart muscle
\item
  \textbf{Sex:} Men are generally at greater risk of heart disease. The
  risk for women increases after menopause.
\item
  \textbf{Family history:} A family history of heart disease increases
  the risk of coronary artery disease, especially if a parent developed
  it at an early age (before age 55 for a male relative, such as ones
  brother or father, and 65 for a female relative, such as ones mother
  or sister).
\item
  \textbf{Smoking:} Nicotine tightens blood vessels, and carbon monoxide
  can damage their inner lining, making them more susceptible to
  atherosclerosis. Heart attacks are more common in smokers than in
  nonsmokers.
\item
  \textbf{Poor diet:} A diet that's high in fat, salt, sugar and
  cholesterol can contribute to the development of heart disease.
\item
  \textbf{High blood pressure:} Uncontrolled high blood pressure can
  result in hardening and thickening of arteries, narrowing the vessels
  through which blood flows.
\item
  \textbf{High blood cholesterol levels:} High levels of cholesterol in
  blood can increase the risk of plaque formation and atherosclerosis.
\item
  \textbf{Diabetes:} Diabetes increases the risk of heart disease. Both
  conditions share similar risk factors, such as obesity and high blood
  pressure.
\item
  \textbf{Obesity:} Excess weight typically worsens other heart disease
  risk factors.
\item
  \textbf{Physical inactivity:} Lack of exercise also is associated with
  many forms of heart disease and some of its other risk factors as
  well.
\item
  \textbf{Stress:} Unrelieved stress may damage the arteries and worsen
  other risk factors for heart disease.
\end{itemize}

\hypertarget{problem-definition}{%
\subparagraph{1.5 PROBLEM DEFINITION:}\label{problem-definition}}

The major challenge in heart disease is its detection. There are
instruments available which can predict heart disease but either they
are expensive or are not efficient to calculate chance of heart disease
in human. Early detection of cardiac diseases can decrease the mortality
rate and overall complications. However, it is not possible to monitor
patients every day in all cases accurately and consultation of a patient
for 24 hours by a doctor is not available since it requires more
sapience,time and expertise. Since we have a good amount of data in
today's world, we can use various statistical analysis to analyze the
data for hidden patterns. The hidden patterns can be used for health
diagnosis in medicinal data.

\hypertarget{objective-of-the-project}{%
\subparagraph{2 OBJECTIVE OF THE
PROJECT:}\label{objective-of-the-project}}

In this project work our objective is to model the chance of occurance
of heart disease based on several associated covariates or features
(age, sex, cholesterol, blood sugar blood pressure etc.).The goal of our
heart disease prediction project is to determine if a patient should be
diagnosed with heart disease or not, which is a binary outcome, so:
Positive result = 1, the patient will be diagnosed with heart disease.
Negative result = 0, the patient will not be diagnosed with heart
disease. We have to find which model has the greatest accuracy and
identify correlations in our data. Finally, we also have to determine
which features are the most influential in our heart disease diagnosis.
This will act as a tool to the physicians for predicting the probability
of heart disease of a patient given the values of the relevant
covariates of him/her

\hypertarget{data-description}{%
\subparagraph{3 DATA DESCRIPTION:}\label{data-description}}

Cardiovascular diseases (CVDs) are the number 1 cause of death globally,
taking an estimated 17.9 million lives each year, which accounts for
31\% of all deaths worldwide. Four out of 5 CVD deaths are due to heart
attacks and strokes, and one-third of these deaths occur prematurely in
people under 70 years of age. Heart failure is a common event caused by
CVDs and this dataset contains 11 features that can be used to predict a
possible heart disease.

\hypertarget{source-of-the-data}{%
\subparagraph{3.1 SOURCE OF THE DATA:}\label{source-of-the-data}}

\begin{itemize}
\tightlist
\item
  \textbf{Dataset:}
  (\url{https://www.kaggle.com/datasets/fedesoriano/heart-failure-prediction})
\end{itemize}

This dataset was created by combining different datasets already
available independently but not combined before. In this dataset, 5
heart datasets are combined over 11 common features . The five datasets
used for its curation are:

\begin{itemize}
\item
  Cleveland: 303 observations
\item
  Hungarian: 294 observations
\item
  Switzerland: 123 observations
\item
  Long Beach VA: 200 observations
\item
  Stalog (Heart) Data Set: 270 observations
\end{itemize}

Total: 1190 observations

Duplicated: 272 observations

Final dataset: 918 observation

3.2 THE DATA:

This dataset contains 11 features age, sex chest pain type, resting
blood pressure, fasting blood sugar, resting ecg, ST-slope, Exercise
angina, Maximum heart rate achieved, cholesterol, oldpeak and the
respons variable is Heart disease. From the original dataset we have
converted the categorical covariates (Sex, Chestpain type, RestingECG,
ExcerciseAgina, ST-slope) and the dependent variable or response heart
disease into numerical form. Here some of the observations are shown in
the following table:

\begin{verbatim}
##    Age Sex ChestPainType RestingBP Cholesterol FastingBS RestingECG MaxHR
## 1   40   M           ATA       140         289         0     Normal   172
## 2   49   F           NAP       160         180         0     Normal   156
## 3   37   M           ATA       130         283         0         ST    98
## 4   48   F           ASY       138         214         0     Normal   108
## 5   54   M           NAP       150         195         0     Normal   122
## 6   39   M           NAP       120         339         0     Normal   170
## 7   45   F           ATA       130         237         0     Normal   170
## 8   54   M           ATA       110         208         0     Normal   142
## 9   37   M           ASY       140         207         0     Normal   130
## 10  48   F           ATA       120         284         0     Normal   120
##    ExerciseAngina Oldpeak ST_Slope HeartDisease
## 1               N     0.0       Up            0
## 2               N     1.0     Flat            1
## 3               N     0.0       Up            0
## 4               Y     1.5     Flat            1
## 5               N     0.0       Up            0
## 6               N     0.0       Up            0
## 7               N     0.0       Up            0
## 8               N     0.0       Up            0
## 9               Y     1.5     Flat            1
## 10              N     0.0       Up            0
\end{verbatim}

\hypertarget{graphical-representation-of-the-data}{%
\paragraph{4 GRAPHICAL REPRESENTATION OF THE
DATA:}\label{graphical-representation-of-the-data}}

\textbf{(i)Presence and absence of heart disease:}

\includegraphics{PROJECT-REPORT_files/figure-latex/unnamed-chunk-4-1.pdf}

FIGURE 5: Presence and absence of Heart Disease.

\textbf{(ii) Age analysis:}

\includegraphics{PROJECT-REPORT_files/figure-latex/unnamed-chunk-5-1.pdf}

\hypertarget{figure-6-age-count}{%
\subparagraph{FIGURE 6: Age count}\label{figure-6-age-count}}

\begin{itemize}
\tightlist
\item
  From the above graph we can say that most of patients are between age
  50 to 65 in the data
\end{itemize}

\textbf{(iii) BP analysis:}

\includegraphics{PROJECT-REPORT_files/figure-latex/unnamed-chunk-6-1.pdf}

\hypertarget{figure-7-bp-count}{%
\subparagraph{FIGURE 7: BP count}\label{figure-7-bp-count}}

\textbf{(iv) Compare Blood pressure across chest pain:}

\includegraphics{PROJECT-REPORT_files/figure-latex/unnamed-chunk-7-1.pdf}

\hypertarget{figure-8-compare-blood-pressure-across-chest-pain}{%
\subparagraph{FIGURE 8: Compare Blood pressure across chest
pain}\label{figure-8-compare-blood-pressure-across-chest-pain}}

\begin{itemize}
\tightlist
\item
  Here we are comparing blood pressure across chest pain for males and
  females. From the above graph we can see that there are some outliers.
\end{itemize}

\textbf{(v) Compare Cholesterol across chest pain:}

\includegraphics{PROJECT-REPORT_files/figure-latex/unnamed-chunk-8-1.pdf}

\hypertarget{figure-9-compare-cholesterol-across-chest-pain}{%
\subparagraph{FIGURE 9: Compare Cholesterol across chest
pain}\label{figure-9-compare-cholesterol-across-chest-pain}}

\begin{itemize}
\tightlist
\item
  Here we are comparing Cholesterol across chest pain for males and
  females. From the above graph we can see that there are some outliers.
\end{itemize}

\textbf{(vi) Correlation between the attributes:}

\includegraphics{PROJECT-REPORT_files/figure-latex/unnamed-chunk-9-1.pdf}

\hypertarget{figure-10-correlation-plot}{%
\subparagraph{FIGURE 10: Correlation
plot}\label{figure-10-correlation-plot}}

\begin{itemize}
\tightlist
\item
  From the above plot we can see that correlation between Age and
  cholesterol;BP and cholesterol; BP and maximum Heart rate; cholesterol
  and oldpeak is very low.
\end{itemize}

\hypertarget{rayyan}{%
\section{rayyan}\label{rayyan}}

\hypertarget{two-by-two-table-analysis}{%
\paragraph{6.3 TWO BY TWO TABLE
ANALYSIS:}\label{two-by-two-table-analysis}}

After preparing data we can do some case control by making sure all of
the factor levels are represented by patients with and without heart
disease and the results are given below:

\emph{TABLE 1}

\hypertarget{sex}{%
\paragraph{SEX}\label{sex}}

\begin{longtable}[]{@{}rrr@{}}
\toprule\noalign{}
HeartDisease & FEMALE & MALE \\
\midrule\noalign{}
\endhead
\bottomrule\noalign{}
\endlastfoot
NO & 143 & 267 \\
YES & 50 & 458 \\
\end{longtable}

\emph{TABLE 2}

\hypertarget{chestpaintype}{%
\paragraph{ChestPainType}\label{chestpaintype}}

\begin{longtable}[]{@{}rrrrr@{}}
\toprule\noalign{}
HeartDisease & ASY & ATA & NAP & TA \\
\midrule\noalign{}
\endhead
\bottomrule\noalign{}
\endlastfoot
0 & 104 & 149 & 131 & 26 \\
1 & 392 & 24 & 72 & 20 \\
\end{longtable}

\emph{TABLE 3}

\hypertarget{fastingbs}{%
\paragraph{FastingBS}\label{fastingbs}}

\begin{longtable}[]{@{}rrr@{}}
\toprule\noalign{}
HeartDisease & \textless=120 & \textgreater120 \\
\midrule\noalign{}
\endhead
\bottomrule\noalign{}
\endlastfoot
NO & 366 & 44 \\
YES & 338 & 170 \\
\end{longtable}

\emph{TABLE 4}

\hypertarget{restingecg}{%
\paragraph{RestingECG}\label{restingecg}}

\begin{longtable}[]{@{}rrrr@{}}
\toprule\noalign{}
HeartDisease & LVH & Normal & ST \\
\midrule\noalign{}
\endhead
\bottomrule\noalign{}
\endlastfoot
0 & 82 & 267 & 61 \\
1 & 106 & 285 & 117 \\
\end{longtable}

\emph{TABLE 5}

\hypertarget{chestpaintype-1}{%
\paragraph{ChestPainType}\label{chestpaintype-1}}

\begin{longtable}[]{@{}rrrrr@{}}
\toprule\noalign{}
HeartDisease & ASY & ATA & NAP & TA \\
\midrule\noalign{}
\endhead
\bottomrule\noalign{}
\endlastfoot
0 & 104 & 149 & 131 & 26 \\
1 & 392 & 24 & 72 & 20 \\
\end{longtable}

\emph{TABLE 6}

\hypertarget{st_slope}{%
\paragraph{ST\_Slope}\label{st_slope}}

\begin{longtable}[]{@{}rrrr@{}}
\toprule\noalign{}
HeartDisease & Down & Flat & Up \\
\midrule\noalign{}
\endhead
\bottomrule\noalign{}
\endlastfoot
0 & 14 & 79 & 317 \\
1 & 49 & 381 & 78 \\
\end{longtable}

\emph{TABLE 7}

\hypertarget{exerciseangina}{%
\paragraph{ExerciseAngina}\label{exerciseangina}}

\begin{longtable}[]{@{}rrr@{}}
\toprule\noalign{}
HeartDisease & N & Y \\
\midrule\noalign{}
\endhead
\bottomrule\noalign{}
\endlastfoot
0 & 355 & 55 \\
1 & 192 & 316 \\
\end{longtable}

\begin{itemize}
\tightlist
\item
  \textbf{Analysis of 2×2 table (TABLE:2):}
\end{itemize}

Looking at the raw data from Table 2 we can say that most of the females
don't have heart disease and most of the males have heart disease. Being
female is likely to decrease the odds in having heart disease. Odds in
case of female patients = 0.3496 In other words, if a sample is female,
the odds are against it that result will be Yes. Being male is likely to
increase the odds in having heart disease. Odds in case of male patients
= 1.7153 In other words, if a sample is male, the odds are for it that
the result will be yes. Here is the result of 2×2 table analysis for
comparison test in R:

\begin{verbatim}
## 2 by 2 table analysis: 
## ------------------------------------------------------ 
## Outcome   : FEMALE 
## Comparing : NO vs. YES 
## 
##     FEMALE MALE    P(FEMALE) 95% conf. interval
## NO     143  267       0.3488    0.3042   0.3962
## YES     50  458       0.0984    0.0754   0.1275
## 
##                                    95% conf. interval
##              Relative Risk: 3.5436    2.6395   4.7574
##          Sample Odds Ratio: 4.9059    3.4378   7.0011
## Conditional MLE Odds Ratio: 4.8971    3.3972   7.1467
##     Probability difference: 0.2504    0.1972   0.3030
## 
##              Exact P-value: 0.0000 
##         Asymptotic P-value: 0.0000 
## ------------------------------------------------------
\end{verbatim}

\hypertarget{analysis-using-logistic-regression}{%
\paragraph{6.4 ANALYSIS USING LOGISTIC
REGRESSION:}\label{analysis-using-logistic-regression}}

Now we will jump to the logistic regression. Let us create a very simple
model that uses sex to predict Heart disease. Let us denote sex with X.
So here the model will be \(ln(\frac{\pi}{1-\pi})=\beta_o+\beta_1X\)

\begin{verbatim}
##              Estimate Std. Error   z value     Pr(>|z|)
## (Intercept) -1.050822  0.1642953 -6.395931 1.595726e-10
## SexMALE      1.590442  0.1814433  8.765503 1.859444e-18
\end{verbatim}

\hypertarget{interpretation}{%
\paragraph{Interpretation:}\label{interpretation}}

Going through the first coefficient, the intercept is the log(odds) a
female has heart disease. This is because female is the first factor in
``sex'' (the factors are ordered, alphabetically by default,``female'',
``male''). Now the second coefficient, sexMALE is the log(odds ratio)
that tells that if a sample has sex=MALE, the odds of having heart
disease are, on a log scale, 1.59 times greater than if a sample has
sex=FEMALE. Now let us see what this logistic regression predicts, given
that a patient is either female or male through a plot;

\includegraphics{PROJECT-REPORT_files/figure-latex/unnamed-chunk-13-1.pdf}

Since there are only two probabilities (one for females and one for
males), we can use a table to summarize the predicted probabilities. The
result is given below:

\begin{verbatim}
##                    Sex
## probability.of.hd   FEMALE MALE
##   0.259067357513526    193    0
##   0.63172413793103       0  725
\end{verbatim}

\hypertarget{table-8-predicted-probabilities-of-getting-heart-disease.}{%
\subsubsection{TABLE 8: Predicted probabilities of getting heart
disease.}\label{table-8-predicted-probabilities-of-getting-heart-disease.}}

Now we will use other available covariates for prediction. Before
creating model, we check our target variable proportions, i.e

\begin{verbatim}
## 
##        NO       YES 
## 0.4466231 0.5533769
\end{verbatim}

\textbf{Selection of covariates:} The common approach to statistical
model building is minimization of variables until the most parsimonious
model model that describes the data which also results in numerical
stability and generalizability of the results. Inclusion of all other
relevant variables in the model regardless of their significance can
lead to numerical unstable estimates. Therefore We have to made
purposeful selection of variables in logistic regression. \textbf{A
decision to keep a variable in the model might be based on the
statistical significance.} Finally, statistical significance should not
be the sole criterion for inclusion of a term in a model. It is sensible
to include a variable that is central to the purposes of the study and
report its estimated effect even if it is not statistically significant.
Other criteria besides significance tests can help select a good model
in terms of estimating quantities of interest. \textbf{The best known is
the Akaike information criterion i.e AIC.} AIC=-2×(maximized log
likelihood - number of parameters in model) .We will use several
variables that may have a significant effect toward our target variable
like, age, sex, blood pressure, cholesterol, maximum heart rate,
oldpeak. So, We will start with age(X1), sex(X2) (Model1). Therefore as
per previous denotation here the equation is,

\(ln \left(\frac{\pi_i}{1-\pi_i} \right) = \sum_{j=0}^{p}\beta_j x_{ij}\)

independent with \(x_{io}\) = 1, i = 1(1)918 and p = 1, 2 Then we
estimate the coefficients, the result given below

\begin{verbatim}
##                Estimate  Std. Error   z value     Pr(>|z|)
## (Intercept) -4.63489288 0.481185190 -9.632243 5.843942e-22
## Age          0.06653121 0.008165056  8.148285 3.691216e-16
## SexM         1.64119501 0.189344877  8.667755 4.407236e-18
\end{verbatim}

\textbf{Interpretation:} We can see that Age and sex are significant
toward heart disease.So keeping them in the model we add blood pressure
and cholesterol (Model2) and perform the previous process. The results
are shown below

\begin{verbatim}
##                 Estimate   Std. Error   z value     Pr(>|z|)
## (Intercept) -4.536556683 0.6714886987 -6.755969 1.418840e-11
## RestingBP    0.008729276 0.0041910009  2.082862 3.726381e-02
## Cholesterol -0.003738244 0.0007424775 -5.034825 4.782861e-07
## Age          0.059802161 0.0084460957  7.080450 1.436868e-12
## SexM         1.477859457 0.1931725726  7.650462 2.002583e-14
\end{verbatim}

From the above result we see that Blood pressure is significant at 5\%
level with heart disease. Now we will add maximum heart rate and old
peak (Model3)

\begin{verbatim}
##                 Estimate   Std. Error    z value     Pr(>|z|)
## (Intercept)  1.172133089 0.9994802310  1.1727426 2.408990e-01
## RestingBP    0.004030559 0.0047297234  0.8521765 3.941161e-01
## Cholesterol -0.003847012 0.0008394794 -4.5826160 4.591949e-06
## Age          0.020919215 0.0099433645  2.1038367 3.539269e-02
## SexM         1.300474066 0.2138253510  6.0819452 1.187332e-09
## MaxHR       -0.026140546 0.0037676281 -6.9381970 3.971357e-12
## Oldpeak      0.946047331 0.0976132382  9.6917933 3.267403e-22
\end{verbatim}

after adding two continuous covariates blood pressure becomes
insignificant, which is unlikely. Therefore we need to drop them from
the model as well as we will drop age (Model4) and then perform logistic
regression again. And here is the results below:

\begin{verbatim}
##                 Estimate   Std. Error   z value     Pr(>|z|)
## (Intercept) -2.225978977 0.5651345868 -3.938848 8.187393e-05
## RestingBP    0.016196485 0.0040280049  4.020970 5.795908e-05
## Cholesterol -0.004181964 0.0007185532 -5.819978 5.885549e-09
## SexM         1.443330294 0.1869281226  7.721312 1.151386e-14
\end{verbatim}

So we can see that Blood pressure is significant with heart disease when
we use cholesterol and sex as other two covariate. Now we illustrate the
AIC values along with residual variance for this four models.

\emph{TABLE 9}

\begin{longtable}[]{@{}rrrr@{}}
\toprule\noalign{}
Model predictors & Residual deviance & df & AIC \\
\midrule\noalign{}
\endhead
\bottomrule\noalign{}
\endlastfoot
Model 1 & 1101.6 & 915 & 1107.6 \\
Model 2 & 1072.5 & 913 & 1082.5 \\
Model 3 & 885.97 & 911 & 899.97 \\
Model 4 & 1126.5 & 914 & 1134.5 \\
\end{longtable}

For the 3rd model the AIC value is smallest. We will use this model for
prediction as \textbf{lower AIC values indicate a better-fit model}

\textbf{Prediction:} Let us predict response variable using 3rd model.
Then we predict the response variable heart disease using those 6
features or covariates and the plot is given below:

\includegraphics{PROJECT-REPORT_files/figure-latex/unnamed-chunk-20-1.pdf}

\hypertarget{conclusion}{%
\subsubsection{6.5 CONCLUSION:}\label{conclusion}}

The main problem in model building situations is to choose from a large
set of covariates those that should be included in the `best' model.
From the simple one covariate model we got the probability of getting
heart disease is more likely for male (63.17\%) than female (25.91\%).
Here we have chosen 6 covariates. We have seen that while taking account
maximum heart rate and oldpeak, blood pressure becomes insignificant
with heart disease which is not expected. In our data the correlation
between heart rate and BP is negative i.e -0.112135. In real life the
relationship between blood pressure and heart rate is location
dependent. Therefore in spite of insignificance of Blood pressure, it
can be used as a feature for prediction. So as per our analysis it is
proposed to use the 6 features, Resting blood pressure, cholesterol,
age, sex, maximum heart rate achieved and Oldpeak to predict heart
disease

\includegraphics{PROJECT-REPORT_files/figure-latex/unnamed-chunk-21-1.pdf}

\textbf{FIGURE 13: Correlation between continuous variables}

\end{document}
